%% Version: 0.2 (31.01.2018)

%% Choose language: english or german
%% Choose Thesis type: seminar, bachelor, master, techreport
%% Use 'declaration' parameter if you want to generate declaration page
%% Use 'final' to disable Todo-notes from final version without deleting each one of them
\documentclass[german,master]{KITthesis}

%% ---------------------------------
%% | Information about the thesis  |
%% ---------------------------------
\title{My Title\\
is Long}
\titleotherlanguage{Mein Titel\\
ist lang}

\author{My Name}
\address{My Address}
\city{7613x Karlsruhe}
\email{my.email@kit.edu}

\keywords{Keywords, of, my, Thesis}
\keywordsotherlanguge{Die, Stichw\"orter, f\"ur, meine, Arbeit}

%% Study program or a seminar/subject
\studyprogram{Intelligente Industrieroboter}

%% Name of your institute (Default: IAR-IPR)
% \institute{}
%% Name of your faculty (Default: KIT-Fakultät für Informatik
% \KITfaculty{}
%% Address of your institute (Default: Engler-Bunte-Ring 8)
% \instituteaddress{}
%% Insitute City (Default: 76131 Karlsruhe)
% \institutecity{}

\reviewerone{Prof. Dr.-Ing. Torsten Kröger}
\reviewertwo{Prof. Dr.-Ing. habil. Björn Hein}
%
% %% The advisors are PhDs or Postdocs
\advisorone{M.Sc. C}
% %% The second advisor can be omitted
\advisortwo{M.Sc. D}
%
% %% Please enter the start end end time of your thesis
\editingtime{xx. Month 20XX}{xx. Month 20XX}

%% --------------------------------
%% | Settings for word separation |
%% --------------------------------
% Help for separation:
% In german package the following hints are additionally available:
% "- = Additional separation
% "| = Suppress ligation and possible separation (e.g. Schaf"|fell)
% "~ = Hyphenation without separation (e.g. bergauf und "~ab)
% "= = Hyphenation with separation before and after
% "" = Separation without a hyphenation (e.g. und/""oder)

% Describe separation hints here:
\hyphenation{
% Pro-to-koll-in-stan-zen
% Ma-na-ge-ment  Netz-werk-ele-men-ten
% Netz-werk Netz-werk-re-ser-vie-rung
% Netz-werk-adap-ter Fein-ju-stier-ung
% Da-ten-strom-spe-zi-fi-ka-tion Pa-ket-rumpf
% Kon-troll-in-stanz
}


%% ------------------------
%% |    Including files   |
%% ------------------------
% Only files listed here will be included!
% Userful command for partially translating the document (for bug-fixing e.g.)
% \includeonly{%
% Content/0-Declaration,
% Content/0-Abstract_EN,
% Content/0-Abstract_DE,
% Content/1-Introduction,
% Content/2-State-of-the-art,
% Content/3-Methods,
% Content/4-Concept,
% Content/5-Implementation,
% Content/6-Results,
% Content/7-Discussion,
% Content/8-Conclusion,
% Content/11-Appendix,
% }

\settitle
%%%%%%%%%%%%%%%%%%%%%%%%%%%%%%%%%
%% Here, main documents begins %%
%%%%%%%%%%%%%%%%%%%%%%%%%%%%%%%%%
\begin{document}
%% Uncomment this to use only first authors name in bibliography
% \bstctlcite{BSTcontrol}

%% Set PDF metadata
\setpdf

%% Title Page
\includetitle

% TODO: Remove this from final version
\includelistoftodos

\includedeclaration

\includeacknowledgments

%% ----------------
%% |   Abstract   |
%% ----------------
%% An abstract both in English
%% and German is mandatory.
%%
%% The text is included from the following files:
%% - Content/0-Abstract_EN
%% - Content/0-Abstract_DE
\includeabstract

%% ------------------------
%% |   Table of Contents  |
%% ------------------------
\inculdetableofcontents

\makenomenclature

%% -----------------
%% |   Main part   |
%% -----------------
\setmainpart

%% ==============================
\chapter{\iflanguage{ngerman}{Einleitung}{Introduction}}
\label{sec:Introduction}
%% ==============================

As an useful aid in all scientific work following book is recommended: \cite{deininger1992studienarbeiten}


\subsection{Inline lists}
My robot can:
\begin{enumerate*}[label=(\roman*)]
 \item forward and backward movements,
 \item sidewards movements,
 \item rotation along any curve in space,
 \item place of artificial forces along paths.
\end{enumerate*}

\begin{enumerate*}[label=(\arabic*),itemjoin={{; }}]
    \item the independently controllable wheels
    \item the rechargeable battery pack
    \item the Sick LMS100 laser range scanner
    \item the force-torque sensor
    \item the handlebar for controlling the robotic device
\end{enumerate*}

\url{https://ctan.math.illinois.edu/macros/latex/contrib/enumitem/enumitem.pdf}

You can use inline comments \textbackslash comment\{text to comment\}

\dots
\todo{Rewrite this section}
\nomenclature{IAR-IPR}{Institute for Anthropomatics and Robotics (IAR) - Intelligent Process Control and Robotics (IPR)}
\Blindtext

\subsection{Introduction Sub}
\todo[color=green]{Stuff}
% \todo[due=2017-08-18]{Stuff}
\Blindtext
% \todo[done]{Stuff}
\Blindtext


\subsection{SI Units}
Please use \texttt{siunitx} package for this. See:  \url{https://ctan.org/pkg/siunitx}

%% ==============================
\chapter{\iflanguage{ngerman}{Stand der Wissenschaft und Technik}{State of the art}}
\label{sec:state_of_the_art}
%% ==============================

\dots



% %% ==============================
% Part is used only in PhD thesis
\part{The ideas}
\chapter{\iflanguage{ngerman}{Methoden}{Methods}}
\label{sec:methods}
%% ==============================

\dots



\include{Content/4-Concept}
%% ==============================
% Part is used only in PhD thesis
\part{The Implementation}
\chapter{\iflanguage{ngerman}{Implementierung}{Implementation}}
\label{sec:implementation}
%% ==============================

\dots
\missingfigure{Please add some figures}



\include{Content/6-Results}
%% ==============================
% Part is used only in PhD thesis
\part{The Evaluation}
\chapter{\iflanguage{ngerman}{Diskussion}{Discussion}}
\label{sec:discussion}
%% ==============================

\dots



\include{Content/8-Conclusion}

%% --------------------
%% |   Bibliography   |
%% --------------------
\Bibliography{Bibliography/thesis}

%% ----------------
%% |   Appendix   |
%% ----------------
% \cleardoublepage
%% appendix.tex
%%

%% ==============================
\Appendix
\label{ch:Appendix}
%% ==============================


\section{First Appendix Section}
\label{sec:app-first-sections}



\begin{figure} [ht]
  \centering
   ein Bild
  \caption{A figure}
  \label{fig:BPMNBeispiela}
\end{figure}


\dots





%% --------------------
%% |   MyBib Entry    |
%% --------------------
\addmybibentry

\end{document}
